\section{Take-aways}\label{Take-aways}

\begin{frame}{Take-aways}
    \begin{itemize}
        \item <1-> \glspl{dvae} are a natural and powerful extension of \glspl{vae} in which the prior expresses the temporal dependency of
        the data sequence.
        \item <2-> Discrete-time \glspl{dvae} use discretized latent priors to encode temporal dynamics. They work best with regularly spaced data.
        \item <3-> Continuous-time \glspl{dvae} posit a stochastic process as prior for the latent variables. This allows 
        additional flexibility and irregularly-sampled data.
        \item <4-> In \gls{gpvae}, the latent prior is a Gaussian Process that can sometimes be expressed as a 
        solution to a linear \gls{sde}. In that case, Kalman-Bucy filtering and smoothing algorithms can speed up 
        computations.
        \item <5-> Stochastic calculus provides a framework for more general continuous-time priors.
        \item <6-> When expressed as solutions to general \gls{sde}, those more general stochastic process priors are used in 
        \glspl{latent sde}.
    \end{itemize}
\end{frame}

\begin{frame}
    \Huge
    \centering
    Thank you for your attention!
\end{frame}