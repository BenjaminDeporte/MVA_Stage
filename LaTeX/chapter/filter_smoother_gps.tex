\chapter{Filtering, Smoothing, and the Kernels Zoo}\label{sec:filter smoother gps}

Equiped with the stochastic calculus basics, we see in this chapter that the filtering and smoothing 
tasks (ie computing posterior probabilities of the latent variables) provides a complete framework 
for the corresponding tasks in \glspl{dvae}.

We also see that, when a Gaussian process can be formulated as the solution of a linear \gls{sde} (ie
 when the kernel function verifies some properties), then the gaussian process regression problem 
 of computing posterior probabilities can be performed by algorithms in linear time.

In this chapter, we consider \glspl{ct-ssm} and \glspl{cd-ssm}. In both cases, the latent variables are 
defined by a (continuous) \gls{sde}. The observations can be defined by a second \gls{sde}, or by a set of 
discrete-time observations.

Formally, the \gls{ct-ssm} is defined by:

\begin{tcolorbox}[colback=blue!5!white,colframe=black!75!black,title=Continuous-Time State Space model]
    \begin{align}
        dZ_t &= F(Z_t, t)dt + L(Z_t,t) dB_t \\
        dX_t &= H(Z_t,t)dt + d\eta_t
    \end{align}
    where:
    \begin{itemize}
        \item $Z_t \in \mathbb{R}^{D}$ is the \textit{state}, ie a stochastic process defining the latent variable.
        \item $B_t \in \mathbb{R}^{S}$ is a Brownian motion with diffusion matrix $Q$.
        \item $F \in \mathbb{R}^{D}$ and $L \in \mathbb{R}^{D \times S}$ are the usual drift and dispersion functions.
        \item $X_t \in \mathbb{R}^{M}$ is the \textit{integrated} measurement (or observation) process.
        \item $H \in \mathbb{R}^{M}$ is the observation/measurement model.
        \item $\eta_t \in \mathbb{R}^{S}$ is a Brownian motion with diffusion matrix $R$.
    \end{itemize}
    NB : the observations are assumed to conditionnally independent of the state, and $B_t, \eta_t$ are 
     assumed independent.
    The observation model is equivalent to:
    \begin{align}
        y_t &= \frac{dX_t}{dt} = H(Z_t,t) + \epsilon_t \\
        \epsilon_t &= \frac{d \eta_t}{dt}
    \end{align}
\end{tcolorbox}

Formally, the \gls{cd-ssm} is defined by:

\begin{tcolorbox}[colback=blue!5!white,colframe=black!75!black,title=Continuous-Discrete State Space model]
    \begin{align}
        dZ_t &= F(Z_t, t)dt + L(Z_t,t) dB_t \\
        x_k &\sim p(x_k \vert z_{t_k})
    \end{align}
    where:
    \begin{itemize}
        \item $Z_t \in \mathbb{R}^{D}$ is the \textit{state}, ie a stochastic process defining the latent variable.
        \item $B_t \in \mathbb{R}^{S}$ is a Brownian motion with diffusion matrix $Q$.
        \item $F \in \mathbb{R}^{D}$ and $L \in \mathbb{R}^{D \times S}$ are the usual drift and dispersion functions.
        \item $x_k$ are the observations taken at \textbf{discrete times $(t_k)_{k=1,...,n}$}
    \end{itemize}
    NB : the observations are assumed to conditionnally independent of the state.
\end{tcolorbox}

We see that the \gls{gpvae} is a specific \gls{cd-ssm}, where the underlying latent stochastic process 
is actually a Gaussian process.

Also, the \gls{ct-ssm} assumes a Gaussian observation model, whereas the \gls{cd-ssm} allows more general 
observation models.

From a vocabulary stand-point, we will use indifferently \textit{state} or \textit{latent variable}, and 
\textit{observation} or \textit{measurement}.

\section{Filtering and Smooting}

\textbf{Filtering} is the problem of determining the posterior probability of the latent $Z_t$ given the 
discrete measurements, ie finding $p(Z_t \vert x_{1:k})$ with $t_k \leq t$. This corresponds to 
determining the generative transition probability $p_{\theta_z}(z_t \vert z_{1:t-1}, x_{1:t-1})$ in our 
\gls{dvae} setting.

\textbf{Smoothing} is the problem of determining the posterior probability of the latent $Z_t$ given 
all known observations, ie finding $p(Z_t \vert x_{1:T})$ for all $t \in [0,T]$. This corresponds to 
determining the inference model $q_{\phi}(z_t \vert z_{1:t-1}, x_{1:T})$ in the \gls{dvae} setting.

In general, close-form solutions can be derived when the latent variables \gls{sde} is linear. In continuous 
time, we get the \textbf{Kalman-Bucy} filter equations, which discretize in the well-known \textbf{Kalman filter}.

\begin{tcolorbox}[colback=blue!5!white,colframe=black!75!black,title=Kalman-Bucy filter]
    
\end{tcolorbox}