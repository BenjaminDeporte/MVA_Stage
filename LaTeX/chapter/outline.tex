\chapter{Report Outline}\label{sec:outline}

\textbf{Part II} covers the mathematical background that is required for most of the report. Section 1 covers the main results regarding stochastic processes and introduces the Brownian motion. (Stochastic Differential Equations are introduced later in Part III). Section 2 recaps the main first definitions and results regarding information theory, its application to stochastic processes, some theoretical results for stationary processes and experimental metrics.

\textbf{Part II} covers discrete-time dynamical Variational AutoEncoders. Section 1 is a recap of graphical models and D-Separation, that is used throughout the report. Section 2 introduces the generic formulation of DVAEs : generative and inference models, variational lower bound. Section 3 presents the first, and most simple model : the Deep Kalman Filter. Section 4 presents the most expressive model, ie the Variationnal RNN. Section 5 is a summary of the take-aways regarding the PyTorch-implementations.

\textbf{Part III} takes us to continuous-time Dynamical VAEs. Section 1 is a recap of the theory regarding Gaussian Processes, that will be at the core of this part. Section 2 is intended to be a self-sustained presentation of the stochastic calculus, where we go quickly over the construction of the Ito's integral to derive the necessary Ito's lemma and subsequent calculation rules for the rest of the report.  Section 3 presents the GP-VAE model. Section 4 is a summary of the take-aways regarding the PyTorch implementation.

\textbf{Part IV} summarizes the experiments : Blembet UK tides, Brownian motion, O.U + Student observation model, cryptos...

\textbf{Part V} is an opening to what could be covered next ! Section 1 presents the link between SDEs and Markovian kernel Gaussian Processes. Section 2 presents some results regarding information theory and stochastic processes.