\chapter{Remerciements}\label{sec:Remerciements}

Avec ce travail sur les VAEs dynamiques se tourne une page siginificative de mon parcours en machine learning.

Un grand merci à Pierre-Alexandre Mattei de l'INRIA, qui a accepté d'etre mon enseignant référent pour cette aventure !
Toujours disponible, bienveillant et patient devant les pires de mes questions, avec des idées et l'expérience 
pour me débloquer aux moments clefs ! Merci également à Pierre Latouche, avec lequel Pierre-Alexandre a assuré 
le cours "Introduction to Probabilistic Graphical Models and Deep Generative Models" au 1er trimestre du MVA.

Merci également à l'équipe du MVA : direction, enseignants, administration. Le Master marche bien, j'y ai appris 
énormément de nouvelles choses et consolidé des connaissances plus anciennes. L'enseignement y est bienveillant sans 
etre laxiste, exigeant et productif. Une belle formation, et à titre personnel une belle aventure Francilienne 
pendant quelques mois ! Note à Laurent Oudre : je parle un peu de séries temporelles dans ce rapport, pas mal de 
calcul stochastique, mais pas un mot sur les maths financières, promis !

A mes collègues de l'IRT : merci Lionel ! Me permettre de cumuler mon poste et le MVA pendant une petite année a 
représenté beaucoup pour moi. Toutes les organisations et tous les managers ne l'auraient pas permis, loin de là. 
Aux collègues du 6e étage du B612 - Greg, Franck, Lucas, Sebastien... : oui ma densité de probabilité a chuté, 
mais elle s'apprete à remonter ! Ne revendez pas tout de suite le mobilier de bureau. A bientot.

A mon épouse Anne-Laure, si jamais elle lit ce rapport pour s'endormir : j'étais sincère quand je disais tenter le MVA en deux ans 
parce que ce serait le plus raisonnable en terme d'emploi du temps. Mais la logique Bayesienne, après tout, est de changer d'avis 
en fonction des données disponibles. Et avouons-le, je n'ai jamais été du genre patient. Merci à toi.

A Jean-Michel Loubes - qui m'a mis le pied à l'étrier il y a quelques temps déjà, pointé vers les bons textes de référence, fait 
rencontrer beaucoup de monde pour valider ce qui, à l'époque, n'était qu'une piste d'évolution professionnelle parmis d'autres. 
Revenir aux maths, et venir au machine learning, aura été un de mes meilleurs moves professionnels. J'ai retrouvé une envie 
intellectuelle que je n'avais plus connue depuis longtemps : merci à toi. 
Merci également de m'avoir accepté en MAPI3 à Paul Sabatier il y a de cela quelque temps maintenant.

J'oublie probablement du monde, et je m'en excuse par avance. Merci en tout cas de m'avoir accompagné.

Place aux VAEs dynamiques.