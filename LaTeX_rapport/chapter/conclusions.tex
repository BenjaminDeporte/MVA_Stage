\chapter{Conclusions and perspectives}\label{sec:Conclusions}

We have seen that \glspl{vae} can be applied to data sequences by assuming a temporal relationship in the prior over latent variables. 
This is the framework of \glspl{dvae}. When the prior over the latent variables is discrete, this formulation produces, for example, 
\gls{dkf} and \gls{vrnn}, which we detailed, coded and tested on two toy datasets.

Assuming a continuous-time prior over the latent variables provides more expressiveness, as it can handle irregularly sampled data. 
A first model is \gls{gpvae} in which the latent prior is a set of \glspl{gp} over the time dimensions. We detailed, coded and tested 
the \gls{gpvae} on the Sprites dataset. If this model is potentially more expressive than its discrete counterparts, it is also 
tricker to train, as the choice of prior reverts to a choice of kernel functions.

Factoring in stochastic calculus opens a broader perspective and a new field of models. \glspl{gp} include stochastic processes 
solutions to linear \glspl{sde} and allow to use linear-time filtering and smoothing algorithms to reduce computation times. 

The Markovian nature of stochastic processes solutions of general \glspl{sde} allows to use \glspl{sde} to formulate latent priors 
that are suitable for filtering and smoothing. This idea of \glspl{neural-sde} builds on \glspl{neural-ode} and points to a 
set of potentially very expressive models.

It's been quite a personal journey... and it continues.